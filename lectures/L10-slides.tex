\documentclass[aspectratio=43]{beamer}

% Text packages to stop warnings
\usepackage{lmodern}
\usepackage{textcomp}
\usepackage{listings}
\usepackage{multirow}

% Themes
\usetheme{Boadilla}
\setbeamertemplate{footline}[page number]{}
\setbeamertemplate{navigation symbols}{}

% Suppress the navigation bar
\beamertemplatenavigationsymbolsempty

\lstset{basicstyle=\scriptsize, frame=single}

\newenvironment{changemargin}[1]{% 
  \begin{list}{}{% 
    \setlength{\topsep}{0pt}% 
    \setlength{\leftmargin}{#1}% 
    \setlength{\rightmargin}{1em}
    \setlength{\listparindent}{\parindent}% 
    \setlength{\itemindent}{\parindent}% 
    \setlength{\parsep}{\parskip}% 
  }% 
  \item[]}{\end{list}} 

\title{Lecture 10---C++ atomics; Compilers and You}
\date{January 26, 2015}

\begin{document}

%%%%%%%%%%%%%%%%%%%%%%%%%%%%%%%%%%%%%%%%%%%%%%%%%%%%%%%%%%%%%%%%%%%%%%%%%%%%%%%%
\begin{frame}[plain]
  \titlepage
\end{frame}
%%%%%%%%%%%%%%%%%%%%%%%%%%%%%%%%%%%%%%%%%%%%%%%%%%%%%%%%%%%%%%%%%%%%%%%%%%%%%%%%

%%%%%%%%%%%%%%%%%%%%%%%%%%%%%%%%%%%%%%%%%%%%%%%%%%%%%%%%%%%%%%%%%%%%%%%%%%%%%%%%
\begin{frame}
  \frametitle{Roadmap}

  \Large
    \begin{changemargin}{2cm}
      Last Time: Synchronization Mechanisms\\
      This Time: C++ atomics; C Compilers; Dependencies
    \end{changemargin}
  
\end{frame}
%%%%%%%%%%%%%%%%%%%%%%%%%%%%%%%%%%%%%%%%%%%%%%%%%%%%%%%%%%%%%%%%%%%%%%%%%%%%%%%%

%%%%%%%%%%%%%%%%%%%%%%%%%%%%%%%%%%%%%%%%%%%%%%%%%%%%%%%%%%%%%%%%%%%%%%%%%%%%%%%%
\part{C++ atomics}
\frame{\partpage}
%%%%%%%%%%%%%%%%%%%%%%%%%%%%%%%%%%%%%%%%%%%%%%%%%%%%%%%%%%%%%%%%%%%%%%%%%%%%%%%%

%%%%%%%%%%%%%%%%%%%%%%%%%%%%%%%%%%%%%%%%%%%%%%%%%%%%%%%%%%%%%%%%%%%%%%%%%%%%%%%%
\begin{frame}[fragile]
  \frametitle{About C++ atomics}
  \begin{changemargin}{2cm}
    You can use the default {\tt std::memory\_order}.\\
    (= sequential consistency)\\[1em]

    \alert{Don't use relaxed atomics unless you're an expert!}\\[1em]
  \end{changemargin}
  \begin{changemargin}{1cm}
    \scriptsize     \url{http://stackoverflow.com/questions/9553591/c-stdatomic-what-is-stdmemory-order-and-how-to-use-them}
  \end{changemargin}
\end{frame}
%%%%%%%%%%%%%%%%%%%%%%%%%%%%%%%%%%%%%%%%%%%%%%%%%%%%%%%%%%%%%%%%%%%%%%%%%%%%%%%%

%%%%%%%%%%%%%%%%%%%%%%%%%%%%%%%%%%%%%%%%%%%%%%%%%%%%%%%%%%%%%%%%%%%%%%%%%%%%%%%%
\begin{frame}[fragile]
  \frametitle{Really, don't use C++ relaxed atomics!}
  \begin{center}
    \includegraphics[width=.8\textwidth]{L10/look_of_disapproval}
  \end{center}
\end{frame}
%%%%%%%%%%%%%%%%%%%%%%%%%%%%%%%%%%%%%%%%%%%%%%%%%%%%%%%%%%%%%%%%%%%%%%%%%%%%%%%%

%%%%%%%%%%%%%%%%%%%%%%%%%%%%%%%%%%%%%%%%%%%%%%%%%%%%%%%%%%%%%%%%%%%%%%%%%%%%%%%%
\begin{frame}[fragile]
  \frametitle{C++ atomics: Key Idea}
  \begin{changemargin}{2cm}
    An \emph{atomic operation} is indivisible.\\[1em]
    Other threads see state before or after the operation,
    nothing in between.
  \end{changemargin}
\end{frame}
%%%%%%%%%%%%%%%%%%%%%%%%%%%%%%%%%%%%%%%%%%%%%%%%%%%%%%%%%%%%%%%%%%%%%%%%%%%%%%%%

%%%%%%%%%%%%%%%%%%%%%%%%%%%%%%%%%%%%%%%%%%%%%%%%%%%%%%%%%%%%%%%%%%%%%%%%%%%%%%%%
\begin{frame}[fragile]
  \frametitle{Simplest: {\tt atomic\_flag}}
  \begin{changemargin}{2cm}
    Represents a boolean flag.\\[1em]
    \begin{lstlisting}
#include <atomic>

atomic_flag f = ATOMIC_FLAG_INIT;
    \end{lstlisting}
  \end{changemargin}
\end{frame}
%%%%%%%%%%%%%%%%%%%%%%%%%%%%%%%%%%%%%%%%%%%%%%%%%%%%%%%%%%%%%%%%%%%%%%%%%%%%%%%%

%%%%%%%%%%%%%%%%%%%%%%%%%%%%%%%%%%%%%%%%%%%%%%%%%%%%%%%%%%%%%%%%%%%%%%%%%%%%%%%%
\begin{frame}[fragile]
  \frametitle{Operations on {\tt atomic\_flag}}
  \begin{changemargin}{2cm}
    Can clear, and can test-and-set:
    \begin{lstlisting}
#include <atomic>

atomic_flag f = ATOMIC_FLAG_INIT;
int foo() {
  f.clear();
  if (f.test_and_set()) {
    // was true
  }
}
    \end{lstlisting}
    ~\\
    \begin{tabbing}
      {\tt test\_and\_set}: \= atomically sets to true, \\
      \> returns previous value.
    \end{tabbing}
    No assignment (=) operator.
  \end{changemargin}
\end{frame}
%%%%%%%%%%%%%%%%%%%%%%%%%%%%%%%%%%%%%%%%%%%%%%%%%%%%%%%%%%%%%%%%%%%%%%%%%%%%%%%%

%%%%%%%%%%%%%%%%%%%%%%%%%%%%%%%%%%%%%%%%%%%%%%%%%%%%%%%%%%%%%%%%%%%%%%%%%%%%%%%%
\begin{frame}[fragile]
  \frametitle{Using more general C++ atomics}

  \begin{changemargin}{1cm}
    Declaring them:
  \end{changemargin}
\begin{lstlisting}
#include <atomic>

atomic<int> x;
\end{lstlisting}
  \begin{changemargin}{1cm}
Libary's implementation: \\
\hspace*{1em} on small types, lock-free operations;\\
\hspace*{1em} on large types, mutexes.
  \end{changemargin}

\end{frame}
%%%%%%%%%%%%%%%%%%%%%%%%%%%%%%%%%%%%%%%%%%%%%%%%%%%%%%%%%%%%%%%%%%%%%%%%%%%%%%%%

%%%%%%%%%%%%%%%%%%%%%%%%%%%%%%%%%%%%%%%%%%%%%%%%%%%%%%%%%%%%%%%%%%%%%%%%%%%%%%%%
\begin{frame}
  \frametitle{What to do with Atomics}

  \begin{changemargin}{2cm}
    \Large
    Kinds of operations:
    \begin{itemize}
    \item reads
    \item writes
    \item read-modify-write (RMW)
    \end{itemize}
  \end{changemargin}
\end{frame}
%%%%%%%%%%%%%%%%%%%%%%%%%%%%%%%%%%%%%%%%%%%%%%%%%%%%%%%%%%%%%%%%%%%%%%%%%%%%%%%%

%%%%%%%%%%%%%%%%%%%%%%%%%%%%%%%%%%%%%%%%%%%%%%%%%%%%%%%%%%%%%%%%%%%%%%%%%%%%%%%%
\begin{frame}[fragile]
  \frametitle{Reads and Writes}

  \begin{changemargin}{2cm}
    C++ has syntax to make these all transparent:
\begin{lstlisting}[language=C++]
#include <atomic>
#include <iostream>

std::atomic<int> ai;
int i;

int main() {
    ai = 4;
    i = ai;
    ai = i;
    std::cout << i;
}
\end{lstlisting}
Can also use {\tt i = ai.load()} and {\tt ai.store(i)}.
  \end{changemargin}
\end{frame}
%%%%%%%%%%%%%%%%%%%%%%%%%%%%%%%%%%%%%%%%%%%%%%%%%%%%%%%%%%%%%%%%%%%%%%%%%%%%%%%%

%%%%%%%%%%%%%%%%%%%%%%%%%%%%%%%%%%%%%%%%%%%%%%%%%%%%%%%%%%%%%%%%%%%%%%%%%%%%%%%%
\begin{frame}
  \frametitle{Read-Modify-Write (RMW)}

  \begin{changemargin}{2cm}
    Consider {\tt ai++}.\\[1em]
    This is really \\
    ~~~{\tt tmp = ai.read(); tmp++; ai.write(tmp); }\\[1em]
    Hardware can do that atomically.\\[1em]
    Other RMWs: {\tt +-, \&=, etc, compare-and-swap}\\[2em]

    {\small
      more info:\\ \url{http://preshing.com/20130618/atomic-vs-non-atomic-operations/}
      }
  \end{changemargin}
\end{frame}
%%%%%%%%%%%%%%%%%%%%%%%%%%%%%%%%%%%%%%%%%%%%%%%%%%%%%%%%%%%%%%%%%%%%%%%%%%%%%%%%


%%%%%%%%%%%%%%%%%%%%%%%%%%%%%%%%%%%%%%%%%%%%%%%%%%%%%%%%%%%%%%%%%%%%%%%%%%%%%%%%
\part{Making C Compilers Work For You}
\frame{\partpage}

%%%%%%%%%%%%%%%%%%%%%%%%%%%%%%%%%%%%%%%%%%%%%%%%%%%%%%%%%%%%%%%%%%%%%%%%%%%%%%%%
\begin{frame}
  \frametitle{Three Address Code}

  \begin{changemargin}{2.5cm}
  \begin{itemize}
    \item An intermediate code used by compilers
      for analysis and optimization.
    \vfill
    \item Statements represent one fundamental operation---we
      can consider each operation \structure{atomic}.
    \vfill
    \item Statements have the form:\\
      $\qquad result := operand_1\:operator\:operand_2$
    \vfill
    \item Useful for reasoning about data races,\\ and easier to read than assembly. \\
            \hspace*{1cm} (separates out memory reads/writes).
  \end{itemize}
  \end{changemargin}
\end{frame}
%%%%%%%%%%%%%%%%%%%%%%%%%%%%%%%%%%%%%%%%%%%%%%%%%%%%%%%%%%%%%%%%%%%%%%%%%%%%%%%%

%%%%%%%%%%%%%%%%%%%%%%%%%%%%%%%%%%%%%%%%%%%%%%%%%%%%%%%%%%%%%%%%%%%%%%%%%%%%%%%%
\begin{frame}
  \frametitle{GIMPLE}

  \begin{changemargin}{2.5cm}
  \begin{itemize}
    \item GIMPLE is the three address code used by {\tt gcc}.
    \vfill
    \item To see the GIMPLE representation of your code use the
      {\tt -fdump-tree-gimple} flag.
    \vfill
    \item To see all of the three address code generated by the compiler use
      {\tt -fdump-tree-all}. You'll probably just be interested in the
      optimized version.
    \vfill
    \item Use GIMPLE to reason about your code at a low level without
      having to read assembly.
  \end{itemize}
  \end{changemargin}
\end{frame}
%%%%%%%%%%%%%%%%%%%%%%%%%%%%%%%%%%%%%%%%%%%%%%%%%%%%%%%%%%%%%%%%%%%%%%%%%%%%%%%%

%%%%%%%%%%%%%%%%%%%%%%%%%%%%%%%%%%%%%%%%%%%%%%%%%%%%%%%%%%%%%%%%%%%%%%%%%%%%%%%%
\begin{frame}
  \frametitle{Live Coding Demo: GIMPLE}
\end{frame}


%%%%%%%%%%%%%%%%%%%%%%%%%%%%%%%%%%%%%%%%%%%%%%%%%%%%%%%%%%%%%%%%%%%%%%%%%%%%%%%%
\begin{frame}[fragile]
  \frametitle{Branch Prediction Hints}

  \begin{changemargin}{1.5cm}
  As seen earlier in class, {\tt gcc} allows you to give branch
  prediction hints by calling this builtin function:

  \begin{center}
    \verb+long __builtin_expect (long exp, long c)+
  \end{center}

  The expected result is that {\tt exp} equals {\tt c}.\\[1em]
  Compiler reorders code \& tells CPU the prediction.
  
  \end{changemargin}
\end{frame}
%%%%%%%%%%%%%%%%%%%%%%%%%%%%%%%%%%%%%%%%%%%%%%%%%%%%%%%%%%%%%%%%%%%%%%%%%%%%%%%%

%%%%%%%%%%%%%%%%%%%%%%%%%%%%%%%%%%%%%%%%%%%%%%%%%%%%%%%%%%%%%%%%%%%%%%%%%%%%%%%%
\begin{frame}
  \frametitle{The {\tt restrict} Keyword}

   \begin{changemargin}{2.5cm}
   A new feature of C99: ``The restrict type qualifier allows programs to
      be written so that translators can produce significantly faster
      executables.''
  \begin{itemize}
    \item To request C99 in {\tt gcc}, use the {\tt -std=c99} flag.
  \end{itemize}
~\\
   {\tt restrict} means: you are promising the
      compiler that the pointer will never \structure{alias} (another pointer
      will not point to the same data) for the lifetime of the pointer.
  \end{changemargin}

\end{frame}
%%%%%%%%%%%%%%%%%%%%%%%%%%%%%%%%%%%%%%%%%%%%%%%%%%%%%%%%%%%%%%%%%%%%%%%%%%%%%%%%

%%%%%%%%%%%%%%%%%%%%%%%%%%%%%%%%%%%%%%%%%%%%%%%%%%%%%%%%%%%%%%%%%%%%%%%%%%%%%%%%
\begin{frame}[fragile]
  \frametitle{Example of {\tt restrict} (1)}

  \begin{changemargin}{2.5cm}
    Pointers declared with {\tt restrict} must
       never point to the same data.
~\\[1em]
    From Wikipedia:

  \begin{lstlisting}
void updatePtrs(int* ptrA, int* ptrB, int* val) {
    *ptrA += *val;
    *ptrB += *val;
}
  \end{lstlisting}
 Would declaring all these pointers as
      {\tt restrict} generate better code?
  \end{changemargin}

\end{frame}
%%%%%%%%%%%%%%%%%%%%%%%%%%%%%%%%%%%%%%%%%%%%%%%%%%%%%%%%%%%%%%%%%%%%%%%%%%%%%%%%

%%%%%%%%%%%%%%%%%%%%%%%%%%%%%%%%%%%%%%%%%%%%%%%%%%%%%%%%%%%%%%%%%%%%%%%%%%%%%%%%
\begin{frame}[fragile]
  \frametitle{Example of {\tt restrict} (2)}

  \begin{changemargin}{2.5cm}
    Let's look at the GIMPLE:

  \begin{lstlisting}[numbers=left]
void updatePtrs(int* ptrA, int* ptrB, int* val) {
 D.1609 = *ptrA;
 D.1610 = *val;
 D.1611 = D.1609 + D.1610;
 *ptrA = D.1611;
 D.1612 = *ptrB;
 D.1610 = *val;
 D.1613 = D.1612 + D.1610;
 *ptrB = D.1613;
}
  \end{lstlisting}

  \begin{itemize}
    \item Could any operation be left out if all the pointers
      didn't overlap?
  \end{itemize}  
  \end{changemargin}

\end{frame}
%%%%%%%%%%%%%%%%%%%%%%%%%%%%%%%%%%%%%%%%%%%%%%%%%%%%%%%%%%%%%%%%%%%%%%%%%%%%%%%%

%%%%%%%%%%%%%%%%%%%%%%%%%%%%%%%%%%%%%%%%%%%%%%%%%%%%%%%%%%%%%%%%%%%%%%%%%%%%%%%%
\begin{frame}[fragile]
  \frametitle{Example of {\tt restrict} (3)}

  \begin{changemargin}{2.5cm}
  \begin{lstlisting}[numbers=left]
void updatePtrs(int* ptrA, int* ptrB, int* val) {
 D.1609 = *ptrA;
 D.1610 = *val;
 D.1611 = D.1609 + D.1610;
 *ptrA = D.1611;
 D.1612 = *ptrB;
 D.1610 = *val;
 D.1613 = D.1612 + D.1610;
 *ptrB = D.1613;
}
  \end{lstlisting}
  \begin{itemize}
    \item If {\tt ptrA} and {\tt val} are not equal, you don't have to
      reload the data on {\bf line 7}.
    \item Otherwise, you would: there might be a call\\{\tt ~~~~updatePtrs(\&x, \&y,
      \&x);}
  \end{itemize}
  \end{changemargin}
\end{frame}
%%%%%%%%%%%%%%%%%%%%%%%%%%%%%%%%%%%%%%%%%%%%%%%%%%%%%%%%%%%%%%%%%%%%%%%%%%%%%%%%

%%%%%%%%%%%%%%%%%%%%%%%%%%%%%%%%%%%%%%%%%%%%%%%%%%%%%%%%%%%%%%%%%%%%%%%%%%%%%%%%
\begin{frame}[fragile]
  \frametitle{Example of {\tt restrict} (4)}
  \begin{changemargin}{2.5cm}
  Hence, this markup allows optimization:
  \begin{lstlisting}
void updatePtrs(int* restrict ptrA, 
                int* restrict ptrB,
                int* restrict val)
  \end{lstlisting}
  Note: you can get the optimization by just declaring {\tt ptrA} and
      {\tt val} as {\tt restrict}; {\tt ptrB} isn't needed for this optimization
  \end{changemargin}
\end{frame}
%%%%%%%%%%%%%%%%%%%%%%%%%%%%%%%%%%%%%%%%%%%%%%%%%%%%%%%%%%%%%%%%%%%%%%%%%%%%%%%%

%%%%%%%%%%%%%%%%%%%%%%%%%%%%%%%%%%%%%%%%%%%%%%%%%%%%%%%%%%%%%%%%%%%%%%%%%%%%%%%%
\begin{frame}[fragile]
  \frametitle{Summary of {\tt restrict}}

  \begin{changemargin}{2.5cm}
  \begin{itemize}
    \item Use {\tt restrict} whenever you know the pointer will not alias
      another pointer (also declared {\tt restrict})
  \end{itemize}

    It's hard for the compiler to infer pointer aliasing information;
    it's easier for you to specify it.\\[1em]

    $\Rightarrow$ compiler can better optimize your code (more perf!)\\[1em]

    Caveat: don't lie to the compiler, or  you will get
      \alert{undefined behaviour}.\\[1em]

    Aside: {\tt restrict} is not the same as {\tt const}. {\tt const} data can still be
      changed through an alias.
  \end{changemargin}
\end{frame}
%%%%%%%%%%%%%%%%%%%%%%%%%%%%%%%%%%%%%%%%%%%%%%%%%%%%%%%%%%%%%%%%%%%%%%%%%%%%%%%%

\end{document}
