\documentclass[11pt]{article}
\usepackage{listings}
\usepackage{tikz}
\usepackage{url}
\usepackage{hyperref}
%\usepackage{algorithm2e}
\usetikzlibrary{arrows,automata,shapes,positioning}
\tikzstyle{block} = [rectangle, draw, fill=blue!20, 
    text width=2.5em, text centered, rounded corners, minimum height=2em]
\tikzstyle{bw} = [rectangle, draw, fill=blue!20, 
    text width=3.5em, text centered, rounded corners, minimum height=2em]

\newcommand{\handout}[5]{
  \noindent
  \begin{center}
  \framebox{
    \vbox{
      \hbox to 5.78in { {\bf ECE459: Programming for Performance } \hfill #2 }
      \vspace{4mm}
      \hbox to 5.78in { {\Large \hfill #5  \hfill} }
      \vspace{2mm}
      \hbox to 5.78in { {\em #3 \hfill #4} }
    }
  }
  \end{center}
  \vspace*{4mm}
}

\lstset{basicstyle=\ttfamily \scriptsize}

\newcommand{\lecture}[4]{\handout{#1}{#2}{#3}{#4}{Lecture #1}}
\topmargin 0pt
\advance \topmargin by -\headheight
\advance \topmargin by -\headsep
\textheight 8.9in
\oddsidemargin 0pt
\evensidemargin \oddsidemargin
\marginparwidth 0.5in
\textwidth 6.5in

\parindent 0in
\parskip 1.5ex
%\renewcommand{\baselinestretch}{1.25}

% http://gurmeet.net/2008/09/20/latex-tips-n-tricks-for-conference-papers/
\newcommand{\squishlist}{
 \begin{list}{$\bullet$}
  { \setlength{\itemsep}{0pt}
     \setlength{\parsep}{3pt}
     \setlength{\topsep}{3pt}
     \setlength{\partopsep}{0pt}
     \setlength{\leftmargin}{1.5em}
     \setlength{\labelwidth}{1em}
     \setlength{\labelsep}{0.5em} } }
\newcommand{\squishlisttwo}{
 \begin{list}{$\bullet$}
  { \setlength{\itemsep}{0pt}
     \setlength{\parsep}{0pt}
    \setlength{\topsep}{0pt}
    \setlength{\partopsep}{0pt}
    \setlength{\leftmargin}{2em}
    \setlength{\labelwidth}{1.5em}
    \setlength{\labelsep}{0.5em} } }
\newcommand{\squishend}{
\end{list}  }
\newtheorem{example}{Example}


\begin{document}

\lecture{11 --- January 28, 2015}{Winter 2015}{Patrick Lam}{version 1}

We talked about C++11 atomics last time. Someone asked about whether
there was a Pthreads equivalent. Nope, not really.

{\tt gcc} supports atomics via extensions: \\
\url{https://gcc.gnu.org/onlinedocs/gcc/_005f_005fatomic-Builtins.html}

OS X has atomics via OS calls: \\
\url{https://developer.apple.com/library/mac/documentation/Cocoa/Conceptual/Multithreading/ThreadSafety/ThreadSafety.html}

etc\ldots

Reference:
\url{http://stackoverflow.com/questions/1130018/unix-portable-atomic-operations}

\section*{Dependencies}
I've said that some computations appear to be ``inherently sequential''. We talked about a
bunch of real-life analogies:

\begin{itemize}
\item must extract bicycle from garage before closing garage door
 
\item must close washing machine door before starting the cycle
 
\item must be called on before answering questions? (sort of)
 
\item students must submit assignment before course staff can mark the assignment
\end{itemize}


Note that, in this lecture, we are going to assume that memory accesses follow the
sequentially consistent memory model. For instance, if you declared all variables
to be C++11 atomics, that would be fine. This reasoning is not guaranteed to work
in the presence of undefined behaviour, which exists when you have data races.

\paragraph{Main Idea.} A \emph{dependency} prevents parallelization
when the computation $XY$ produces a different result from the
computation $YX$.

\paragraph{Loop- and Memory-Carried Dependencies.} We distinguish
between \emph{loop-carried} and \emph{memory-carried} dependencies.
In a loop-carried dependency, an iteration depends on the result of
the previous iteration. For instance, consider this code to
compute whether a complex number $x_0 + iy_0$ belongs to the Mandelbrot
set.

{\small \begin{verbatim}
    // Repeatedly square input, return number of iterations before 
    // absolute value exceeds 4, or 1000, whichever is smaller.
    int inMandelbrot(double x0, double y0) {
      int iterations = 0;
      double x = x0, y = y0, x2 = x*x, y2 = y*y;
      while ((x2+y2 < 4) && (iterations < 1000)) {
        y = 2*x*y + y0;
        x = x2 - y2 + x0;
        x2 = x*x; y2 = y*y;
        iterations++;
      }
      return iterations;
    }
\end{verbatim} }
In this case, it's impossible to parallelize loop iterations, because
each iteration \emph{depends} on the $(x, y)$ values calculated in the
previous iteration. For any particular $x_0 + iy_0$, you have to run the
loop iterations sequentially.

Note that you can parallelize the Mandelbrot set calculation
by computing the result simultaneously over many points at
once. Indeed, that is a classic ``embarassingly parallel'' problem,
because the you can compute the result for all of the points
simultaneously, with no need to communicate.

On the other hand, a memory-carried dependency is one where the result
of a computation \emph{depends} on the order in which two memory accesses
occur. For instance:

{\small \begin{verbatim}
    int val = 0;

    void g() { val = 1; }
    void h() { val = val + 2; }
\end{verbatim} }

{\sf What are the possible outcomes after executing {\tt g()} and {\tt h()}
in parallel threads?} \\[1em]

\subsection*{RAW, WAR, WAW and RAR}
The most obvious case of a dependency is as follows:
{\small \begin{verbatim}
    int y = f(x);
    int z = g(y);
\end{verbatim} }
This is a read-after-write (RAW), or ``true'' dependency: the first
statement writes {\tt y} and the second statement reads it.
Other types of dependencies are:

\begin{center}
\begin{tabular}{l|ll}
& {\bf Read 2nd} & {\bf Write 2nd} \\ \hline
{\bf Read 1st} & Read after read (RAR) & Write after read (WAR) \\
& No dependency & Antidependency \\
{\bf Write 1st} & Read after write (RAW) & Write after write (WAW) \\
& True dependency & Output dependency
\end{tabular}
\end{center}

The no-dependency case (RAR) is clear. Declaring data immutable 
in your program is a good way to ensure no dependencies.

Let's look at an antidependency (WAR) example.

{\small \begin{center}
\begin{tabular}{ll}
\begin{minipage}{.4\textwidth}
\begin{verbatim}
void antiDependency(int z) {
  int y = f(x);
  x = z + 1;
}
\end{verbatim}
\end{minipage} &
\begin{minipage}{.4\textwidth}
\begin{verbatim}
void fixedAntiDependency(int z) {
  int x_copy = x;
  int y = f(x_copy);
  x = z + 1;
}
\end{verbatim}
\end{minipage} 
\end{tabular}
\end{center} }
{\sf Why is there a problem?}\\[2em]

Finally, WAWs can also inhibit parallelization:

{\small \begin{center}
\begin{tabular}{ll}
\begin{minipage}{.45\textwidth}
\begin{verbatim}
void outputDependency(int x, int z) {
  y = x + 1;
  y = z + 1;
}
\end{verbatim}
\end{minipage} &
\begin{minipage}{.4\textwidth}
\begin{verbatim}
void fixedOutputDependency(int x, int z) {
  y_copy = x + 1;
  y = z + 1;
}
\end{verbatim}
\end{minipage} 
\end{tabular}
\end{center} }

In both of these cases, renaming or copying data can
eliminate the dependence and enable parallelization. Of course,
copying data also takes time and uses cache, so it's not free. One
might change the output locations of both statements and then copy in
the correct output. These are usually more useful when it's not just one
access, but some sort of longer computation.

\section*{Loop-carried Dependencies}
As we said last time, a loop-carried dependency is one where an
iteration depends on the result of the previous iteration. Let's look
at a couple of examples.

\begin{example}
Initially, {\tt a[0]} and {\tt a[1]} are 1.
Can we run these lines in parallel?
\end{example}
\begin{lstlisting}
     a[4] = a[0] + 1;
     a[5] = a[1] + 2;
\end{lstlisting}

\url{http://www.youtube.com/watch?v=jjXyqcx-mYY}. (This one is legit! Really!)

It turns out that there are no dependencies between the two lines. But this is
an atypical use of arrays. Let's look at more typical uses.

\begin{example}
What about this? (Again, all elements initially 1.)
\end{example}
\begin{lstlisting}
    for (int i = 1; i < 12; ++i)
        a[i] = a[i-1] + 1;
\end{lstlisting}

Nope! We can unroll the first two iterations:
\begin{lstlisting}
    a[1] = a[0] + 1
    a[2] = a[1] + 1
\end{lstlisting}

Depending on the execution order, either {\tt a[2] = 3} or {\tt a[2] =
  2}.  In fact, no out-of-order execution here is safe---statements depend
on previous loop iterations, which exemplifies the notion of a
\emph{loop-carried dependency}. You would have
to play more complicated games to parallelize this.

\begin{example}
  Now consider this example---is it parallelizable? (Again, all elements initially 1.)
\end{example}
\begin{lstlisting}
    for (int i = 4; i < 12; ++i)
        a[i] = a[i-4] + 1;
\end{lstlisting}

Yes, to a degree. We can execute 4 statements in parallel at a time:
\begin{itemize}
  \item a[4] = a[0] + 1, a[8] = a[4] + 1
  \item a[5] = a[1] + 1, a[9] = a[5] + 1
  \item a[6] = a[2] + 1, a[10] = a[6] + 1
  \item a[7] = a[3] + 1, a[11] = a[7] + 1
\end{itemize}  
We can say that the array accesses have stride 4---there are no
dependencies between adjacent array elements. In general, consider
dependencies between iterations.

\paragraph{Larger loop-carried dependency example.}
Now consider the following function.

\begin{lstlisting}
  // Repeatedly square input, return number of iterations before 
  // absolute value exceeds 4, or 1000, whichever is smaller.
  int inMandelbrot(double x0, double y0) {
    int iterations = 0;
    double x = x0, y = y0, x2 = x*x, y2 = y*y;
    while ((x2+y2 < 4) && (iterations < 1000)) {
      y = 2*x*y + y0;
      x = x2 - y2 + x0;
      x2 = x*x; y2 = y*y;
      iterations++;
    }
    return iterations;
  }
\end{lstlisting}

How do we parallelize this?

Well, that's a trick question. There's not much that you can do with that
function. What you can do is to run this function sequentially for each
point, and parallelize along the different points.

As mentioned in class, but one potential problem with that
approach is that one point may take disproportionately long. The safe thing to do is to parcel out the work
at a finer granularity. There are
(unsafe!) techniques for dealing with that too. We'll talk about that
later.

What I did do in class was to actually live-code a parallelization
of the Mandelbrot code. I had to first refactor the code, creating
an array to hold the output. Then I added a struct to pass the offset and
stride to the thread. Finally, I invoked pthread to create and join threads.

\end{document}
