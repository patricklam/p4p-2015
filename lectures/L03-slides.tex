\documentclass[aspectratio=43]{beamer}

\usetheme{Boadilla}
\setbeamertemplate{footline}[page number]{}
\setbeamertemplate{navigation symbols}{}

\usepackage{times}
\usepackage{tikz}
\usefonttheme{structurebold}
\usepackage{listings}

% Suppress the navigation bar
\beamertemplatenavigationsymbolsempty

\newenvironment{changemargin}[1]{% 
  \begin{list}{}{% 
    \setlength{\topsep}{0pt}% 
    \setlength{\leftmargin}{#1}% 
    \setlength{\rightmargin}{1em}
    \setlength{\listparindent}{\parindent}% 
    \setlength{\itemindent}{\parindent}% 
    \setlength{\parsep}{\parskip}% 
  }% 
  \item[]}{\end{list}} 

\title{Lecture 03---Gustafson's Law, Concurrency~vs~Parallelism}
\subtitle{ECE 459: Programming for Performance}
\author{Patrick Lam}
\institute{University of Waterloo}
\date{January 9, 2015}

\begin{document}

%%%%%%%%%%%%%%%%%%%%%%%%%%%%%%%%%%%%%%%%%%%%%%%%%%%%%%%%%%%%%%%%%%%%%%%%%%%%%%%%
\begin{frame}[plain]
  \titlepage
\end{frame}
%%%%%%%%%%%%%%%%%%%%%%%%%%%%%%%%%%%%%%%%%%%%%%%%%%%%%%%%%%%%%%%%%%%%%%%%%%%%%%%%

\part{Limitations to Parallelism}

%%%%%%%%%%%%%%%%%%%%%%%%%%%%%%%%%%%%%%%%%%%%%%%%%%%%%%%%%%%%%%%%%%%%%%%%%%%%%%%%
\begin{frame}
  \frametitle{Multiprocessing in Three Acts}

  \begin{changemargin}{2cm}
    Act 1: {\bf Attack of the Clones}\\
    \hspace*{1em} (symmetric multiprocessing systems) \\[1em]
    Act 2: {\bf The Empire Strikes Back}\\
    \hspace*{1em} (Amdahl's Law) \\[1em]
    Act 3: {\bf A New Hope}\\
    \hspace*{1em} (Gustafson's Law)
  \end{changemargin}
\end{frame}
%%%%%%%%%%%%%%%%%%%%%%%%%%%%%%%%%%%%%%%%%%%%%%%%%%%%%%%%%%%%%%%%%%%%%%%%%%%%%%%%

\section{Amdahl's Law: Wrap-up}
%%%%%%%%%%%%%%%%%%%%%%%%%%%%%%%%%%%%%%%%%%%%%%%%%%%%%%%%%%%%%%%%%%%%%%%%%%%%%%%%
\begin{frame}
  \frametitle{Amdahl's Law Generalization}

  \hspace*{2em} \begin{minipage}{.8\textwidth} The program may have many parts, each of which we can tune to
  a different degree.\\[1em]

  Let's generalize Amdahl's Law.\\[1em]

  $f_1, f_2, \ldots, f_n$: fraction of time in part $n$

  $S_{f_1}, S_{f_n}, \ldots, S_{f_n}$: speedup for part $n$
\end{minipage}
  \vfill
  \begin{center}
  \structure{\[\mbox{\em speedup} = \frac{1}{\frac{f_1}{S_{f_1}} + \frac{f_2}{S_{f_2}} + \ldots +
    \frac{f_n}{S_{f_n}}}\]}
  \end{center}
\end{frame}
%%%%%%%%%%%%%%%%%%%%%%%%%%%%%%%%%%%%%%%%%%%%%%%%%%%%%%%%%%%%%%%%%%%%%%%%%%%%%%%%

%%%%%%%%%%%%%%%%%%%%%%%%%%%%%%%%%%%%%%%%%%%%%%%%%%%%%%%%%%%%%%%%%%%%%%%%%%%%%%%%
\begin{frame}
  \frametitle{Application (1)}

  \hspace*{2em} 
  \begin{minipage}{.8\textwidth} Consider a program with 4 parts in the following scenario:\\[2em]
    \begin{tabular}{r|r|r|r}
    \multicolumn{2}{l}{} & \multicolumn{2}{c}{Speedup} \\
    Part & Fraction of Runtime & Option 1 & Option 2\\
    \hline
    1 & 0.55 & 1  & 2\\
    2 & 0.25 & 5  & 1\\
    3 & 0.15 & 3  & 1\\
    4 & 0.05  & 10 & 1\\
  \end{tabular}

~\\[2em]
  We can implement either Option 1 or Option 2. \\
  Which option is better?
  \end{minipage}
\end{frame}
%%%%%%%%%%%%%%%%%%%%%%%%%%%%%%%%%%%%%%%%%%%%%%%%%%%%%%%%%%%%%%%%%%%%%%%%%%%%%%%%

%%%%%%%%%%%%%%%%%%%%%%%%%%%%%%%%%%%%%%%%%%%%%%%%%%%%%%%%%%%%%%%%%%%%%%%%%%%%%%%%
\begin{frame}
  \frametitle{Application (2)}

\begin{minipage}{.8\textwidth}
  \hspace*{2em} ``Plug and chug'' the numbers:\\[1em]
\hspace*{.2\textwidth}\begin{minipage}{.8\textwidth}
  {\bf Option 1} \vspace*{-2em}

  \[ \mbox{\em speedup} = \frac{1}{0.55 + \frac{0.25}{5} + \frac{0.15}{3} + \frac{0.05}{5}}
    = 1.53  \]
~\\[1em]
  {\bf Option 2} \vspace*{-2em}

   \[\mbox{\em speedup} = \frac{1}{\frac{0.55}{2} + 0.45} = 1.38 \hspace*{5.5em}\]
\end{minipage}
\end{minipage}

\end{frame}
%%%%%%%%%%%%%%%%%%%%%%%%%%%%%%%%%%%%%%%%%%%%%%%%%%%%%%%%%%%%%%%%%%%%%%%%%%%%%%%%

%%%%%%%%%%%%%%%%%%%%%%%%%%%%%%%%%%%%%%%%%%%%%%%%%%%%%%%%%%%%%%%%%%%%%%%%%%%%%%%%
\begin{frame}
  \frametitle{Empirically estimating parallel speedup P}

  \hspace*{2em} Useful to know, don't have to commit to memory:
  \vfill
  \[P_{\mbox{\scriptsize estimated}} = \frac{\frac{1}{\mbox{\em speedup}}-1}{\frac{1}{N}-1}\]
  \vfill
  \hspace*{2em} \begin{minipage}{.8\textwidth} \begin{itemize}
    \item Quick way to guess the fraction of parallel code
    \item Use $P_{\mbox{\scriptsize estimated}}$ to predict speedup for a different number of processors
  \end{itemize} \end{minipage}
\end{frame}
%%%%%%%%%%%%%%%%%%%%%%%%%%%%%%%%%%%%%%%%%%%%%%%%%%%%%%%%%%%%%%%%%%%%%%%%%%%%%%%%

%%%%%%%%%%%%%%%%%%%%%%%%%%%%%%%%%%%%%%%%%%%%%%%%%%%%%%%%%%%%%%%%%%%%%%%%%%%%%%%%
\begin{frame}
  \frametitle{Summary of Amdahl's Law}

\hspace*{2em} Important to focus on the part of the program with most impact.\\[1em]

\hspace*{2em} Amdahl's Law:\\[1em]
\hspace*{2em} \begin{minipage}{.8\textwidth}
  \begin{itemize}
    \item estimates perfect performance gains from
          parallelization (under assumptions); but,
    \vfill
    \item only applies to solving a \structure{fixed problem size} in the
          shortest possible period of time
  \end{itemize} \end{minipage}
\end{frame}
%%%%%%%%%%%%%%%%%%%%%%%%%%%%%%%%%%%%%%%%%%%%%%%%%%%%%%%%%%%%%%%%%%%%%%%%%%%%%%%%


\section{Gustafson's Law}
%%%%%%%%%%%%%%%%%%%%%%%%%%%%%%%%%%%%%%%%%%%%%%%%%%%%%%%%%%%%%%%%%%%%%%%%%%%%%%%%
\begin{frame}
  \frametitle{Gustafson's Law: Formulation}

\hspace*{2em} \begin{minipage}{.8\textwidth}
\begin{tabbing}
  $n$:~~~~~~ \= problem size\\[.1em]

  $S(n)$: \> fraction of serial runtime for a parallel execution\\[.1em]

  $P(n)$: \> fraction of parallel runtime for a parallel execution\\
\end{tabbing}
\end{minipage}

\begin{eqnarray*}
T_p &=& S(n) + P(n) = 1 \\
T_s &=& S(n) + N \cdot P(n) 
\end{eqnarray*}

\[ \mbox{\em speedup} = \frac{T_s}{T_p} \]
\end{frame}
%%%%%%%%%%%%%%%%%%%%%%%%%%%%%%%%%%%%%%%%%%%%%%%%%%%%%%%%%%%%%%%%%%%%%%%%%%%%%%%%

%%%%%%%%%%%%%%%%%%%%%%%%%%%%%%%%%%%%%%%%%%%%%%%%%%%%%%%%%%%%%%%%%%%%%%%%%%%%%%%%
\begin{frame}
  \frametitle{Gustafson's Law}

\hspace*{2em}\begin{minipage}{.8\textwidth}
  \structure{$speedup = S(n) + N \cdot P(n)$}\\[1em]

  Assuming the fraction of runtime in serial part decreases as $n$ increases,
  the speedup approaches $N$.

  \begin{itemize}
    \item Yes! Large problems can be efficiently parallelized. (Ask Google.)
  \end {itemize}
\end{minipage}
\end{frame}
%%%%%%%%%%%%%%%%%%%%%%%%%%%%%%%%%%%%%%%%%%%%%%%%%%%%%%%%%%%%%%%%%%%%%%%%%%%%%%%%

\section{}
%%%%%%%%%%%%%%%%%%%%%%%%%%%%%%%%%%%%%%%%%%%%%%%%%%%%%%%%%%%%%%%%%%%%%%%%%%%%%%%%
\begin{frame}
  \frametitle{Driving Metaphor}

  {\bf Amdahl's Law}
  
  Suppose you're travelling between 2 cities 90~km apart. If you travel for an
  hour at a constant speed less than 90 km/h, your average will never equal
  90~km/h, even if you energize after that hour.
  \vfill
  {\bf Gustafson's Law}

  Suppose you've been travelling at a constant speed less than 90~km/h. Given
  enough distance, you can bring your average up to 90~km/h.
\end{frame}
%%%%%%%%%%%%%%%%%%%%%%%%%%%%%%%%%%%%%%%%%%%%%%%%%%%%%%%%%%%%%%%%%%%%%%%%%%%%%%%%


\part{Parallelism versus Concurrency}
\begin{frame}
 
  \partpage
\end{frame}

%%%%%%%%%%%%%%%%%%%%%%%%%%%%%%%%%%%%%%%%%%%%%%%%%%%%%%%%%%%%%%%%%%%%%%%%%%%%%%%%
\begin{frame}
  \frametitle{Parallelism versus Concurrency}

\begin{changemargin}{2cm}
  {\bf Parallelism}

  Two or more tasks are \structure{parallel}\\ \hspace*{2em} if they are running at the same time. 

  Main goal: run tasks as fast as possible. 

  Main concern: \structure{dependencies}.
  \vfill
  {\bf Concurrency}

  Two or more tasks are \structure{concurrent}\\ \hspace*{2em} if the ordering of the two tasks is not 
  predetermined. 

  Main concern: \structure{synchronization}.
\end{changemargin}

\end{frame}
%%%%%%%%%%%%%%%%%%%%%%%%%%%%%%%%%%%%%%%%%%%%%%%%%%%%%%%%%%%%%%%%%%%%%%%%%%%%%%%%

\part{Threads}
\begin{frame}
  \partpage
\end{frame}
%%%%%%%%%%%%%%%%%%%%%%%%%%%%%%%%%%%%%%%%%%%%%%%%%%%%%%%%%%%%%%%%%%%%%%%%%%%%%%%%
\begin{frame}
  \frametitle{Threads}

  \begin{center}
    \includegraphics[scale=0.1]{L03/thread}
  \end{center}

  \begin{changemargin}{2.5cm}
  \begin{itemize}
    \item What are they?
    \vfill
    \item How do operating systems implement them?
    \vfill
    \item How can we leverage them?
  \end{itemize}
  \end{changemargin}
\end{frame}
%%%%%%%%%%%%%%%%%%%%%%%%%%%%%%%%%%%%%%%%%%%%%%%%%%%%%%%%%%%%%%%%%%%%%%%%%%%%%%%%

%%%%%%%%%%%%%%%%%%%%%%%%%%%%%%%%%%%%%%%%%%%%%%%%%%%%%%%%%%%%%%%%%%%%%%%%%%%%%%%%
\begin{frame}
  \frametitle{Processes versus Threads}

\begin{changemargin}{2.5cm}
  {\bf Process}

  An instance of a computer program that contains program code and its:

  \begin{itemize}
    \item Own address space / virtual memory;
    \item Own stack / registers;
    \item Own resources (file handles, etc.).
  \end{itemize}
  \vfill
  {\bf Thread}

  ``Lightweight processes''. 

  In most cases, a thread is contained within a process.

  \begin{itemize}
    \item Same address space as parent process
      \begin{itemize}
        \item Shares access to code and variables with parent.
      \end{itemize}
    \item Own stack / registers
    \item Own thread-specific data
  \end{itemize}

\end{changemargin}

\end{frame}
%%%%%%%%%%%%%%%%%%%%%%%%%%%%%%%%%%%%%%%%%%%%%%%%%%%%%%%%%%%%%%%%%%%%%%%%%%%%%%%%

%%%%%%%%%%%%%%%%%%%%%%%%%%%%%%%%%%%%%%%%%%%%%%%%%%%%%%%%%%%%%%%%%%%%%%%%%%%%%%%%
\begin{frame}
  \frametitle{Software and Hardware Threads}

\begin{changemargin}{2.5cm}
  {\bf Software Thread:}

  What you program with (e.g. with {\tt pthread\_create()} or {\tt std::thread()}). \vfill

  Corresponds to a stream of instructions executed by the processor. \vfill

  On a single-core, single-processor machine, someone has to multiplex the CPU to
  execute multiple threads concurrently; only one thread runs at a time.

  \vfill
  {\bf Hardware Thread:}

  Corresponds to virtual (or real) CPUs in a system. Also known as strands. \vfill

  Operating system must multiplex software threads onto hardware threads, but 
  can execute more than one software thread at once.

\end{changemargin}

\end{frame}
%%%%%%%%%%%%%%%%%%%%%%%%%%%%%%%%%%%%%%%%%%%%%%%%%%%%%%%%%%%%%%%%%%%%%%%%%%%%%%%%


%%%%%%%%%%%%%%%%%%%%%%%%%%%%%%%%%%%%%%%%%%%%%%%%%%%%%%%%%%%%%%%%%%%%%%%%%%%%%%%%
\begin{frame}
  \frametitle{Thread Model---1:1 (Kernel-level Threading)}

    \begin{changemargin}{1.5cm}

    Simplest possible threading implementation.
    \vfill
    The kernel schedules threads on different processors;
      \begin{itemize}
        \item NB: Kernel involvement required to take advantage of a multicore system.
      \end{itemize}
    \vfill
    Context switching involves system call overhead.
    \vfill
    Used by Win32, POSIX threads for Windows and Linux.
    \vfill
    Allows concurrency and parallelism.

    \end{changemargin}
\end{frame}
%%%%%%%%%%%%%%%%%%%%%%%%%%%%%%%%%%%%%%%%%%%%%%%%%%%%%%%%%%%%%%%%%%%%%%%%%%%%%%%%

%%%%%%%%%%%%%%%%%%%%%%%%%%%%%%%%%%%%%%%%%%%%%%%%%%%%%%%%%%%%%%%%%%%%%%%%%%%%%%%%
\begin{frame}
  \frametitle{Thread Model---N:1 (User-level Threading)}

  \begin{changemargin}{1.5cm}
    All application threads map to a single kernel thread.
    \vfill
    Quick context switches, no need for system call.
    \vfill
    Cannot use multiple processors, only for concurrency.
      \begin{itemize}
        \item Why would you use user threads?
      \end{itemize}
    \vfill
    Used by GNU Portable Threads.
  \end{changemargin}
\end{frame}
%%%%%%%%%%%%%%%%%%%%%%%%%%%%%%%%%%%%%%%%%%%%%%%%%%%%%%%%%%%%%%%%%%%%%%%%%%%%%%%%

%%%%%%%%%%%%%%%%%%%%%%%%%%%%%%%%%%%%%%%%%%%%%%%%%%%%%%%%%%%%%%%%%%%%%%%%%%%%%%%%
\begin{frame}
  \frametitle{Thread Model---M:N (Hybrid Threading)}

  \begin{changemargin}{1.5cm}
    Map $M$ application threads to $N$ kernel threads.
    \vfill
    A compromise between the previous two models.
    \vfill
    Allows quick context switches and the use of multiple processors.
    \vfill
    Requires increased complexity:
    \begin{itemize}
        \item Both library and kernel must schedule.
        \item Schedulers may not coordinate well together.
        \item Increases likelihood of priority inversion\\ \hspace*{2em} (recall from Operating Systems).
      \end{itemize}
    \vfill
    Used by Windows 7 threads.
  \end{changemargin}
\end{frame}
%%%%%%%%%%%%%%%%%%%%%%%%%%%%%%%%%%%%%%%%%%%%%%%%%%%%%%%%%%%%%%%%%%%%%%%%%%%%%%%%

%%%%%%%%%%%%%%%%%%%%%%%%%%%%%%%%%%%%%%%%%%%%%%%%%%%%%%%%%%%%%%%%%%%%%%%%%%%%%%%%
\begin{frame}
  \frametitle{Live Coding Demo: Deducing the Thread Model}

  \begin{changemargin}{2cm}
    {\tt time}, {\tt top}, and {\tt perf stat} all let you figure this out.
  \end{changemargin}

\end{frame}
%%%%%%%%%%%%%%%%%%%%%%%%%%%%%%%%%%%%%%%%%%%%%%%%%%%%%%%%%%%%%%%%%%%%%%%%%%%%%%%%

\end{document}
